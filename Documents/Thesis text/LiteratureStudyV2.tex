\chapter{Literature Study}


\section{Methodology}

The systematic search for different approaches happened on a combination of Google Scholar, ACM Digital Library and Web of Science. Using the following keywords and combinations as well as variations "Acoustic Event Detection", "Acoustic Scene Classification", "Multi-task Speech Classification", "Deep learning Audio", "Joint learning acoustic", "Multi-task audio classification", "Multi-task learning", "Multi-task relations", "When is Multi-task effective", a collection of papers was found which was expanded by referenced and referencing papers about Deep learning audio tasks in a multi-task framework, as well as general research on when multi-task learning is effective. The original focus was pure on AED, which expanded into its combinations with other, unrelated audio classification tasks. Special attention was given to papers coming from the DCASE audio classification competitions, a yearly event where a number of different audio classification goals are defined with both returning as well as varying tasks. 

\section{Deep Learning Audio classification}

% Gucci en zo, maar expensive to run
% Voornaamste domeinen zijn Environmental, Speech en Muziek
% Elk domein heeft een verschil in noise patterns, wat hetzelfde algoritme en feature extraction gebruiken moeilijk maakt
% Verschil tussen event-based en context-based i.e. frame level of fragment level
% Verschil tussen mono-phonic en poly-phonic (overlapping v non overlapping)
% DNN, CNN, RNN, CRNN

A growing amount of 

\section{Multi-task Learning}

\section{The Interest in Multi-task Learning in Audio Classification}