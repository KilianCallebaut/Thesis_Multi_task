\chapter{Literature Study}

%
%Should be mentioned
%\begin{itemize}
%	\item	The dataset
%	\begin{itemize}
%		\item Its stats (How many instances per event)
%		\item What is in it
%	\end{itemize} 
%	\item Mention difference in weak and strong labeled training. If weak, label is available, but not the exact temporal boundaries of the event.
%\end{itemize}
%
\section{Sensor-based Lifelog Modeling}

In the first part of the literature review, an analysis will be made of how the literature defines and approaches sensor-based lifelogging tasks. This will have the purpose of finding out what is required for an acoustic event detection, get a clear understanding of what lifelog modeling actually entails and to see which approaches to sensor-based lifelogging have been devised. 

\subsection{Methodology}

The systematic search for different approaches happened on a combination of Google Scholar, ACM Digital Library and Web of Science. Using the following keywords and combinations "lifelog", "lifelogging", "sensor lifelog", "ambient sensor lifelog", "acoustic lifelog", "audio lifelog", "smartphone lifelog", "mobile lifelog", "lifelog modeling" and "lifelog survey", a collection of papers was found which was expanded by referenced and referencing papers about lifelogging. The focus lay on collecting papers about sensor-based automatic lifelog models as well as lifelogging on mobile devices. The research question that is being dealt with is "How can a multi-task acoustic event detection system be developed for lifelog modeling in an office environment?", for which we need to answer what sensor-based lifelogging is defined, what approaches there are, how the sensor information is used and what requirements there are for sensor information extraction systems.

\subsection{Notes}

\begin{itemize}
	\item \cite{rawassizadeh2013ubiqlog} listed 5 design requirements for lifelogging systems:
	\begin{itemize}
		\item seamless integration into daily life (nope)
		\item resource efficiency (yup, through analysis)
		\item security (nope, but can be discussed)
		\item long-term digital preservation (nope, but can be discussed)
		\item information retrieval (needs investigation -> efficient tagging/indexing coming from system)
	\end{itemize}
	\item \cite{rawassizadeh2013ubiqlog} enriched data with semantic info straight on the mobile phone
	\item \cite{bayindir2017survey} mentions a good lifelogging system should operate during the day without user attention, which utilizes phone resources, which has an effect on power consumption.
	\item \cite{bayindir2017survey} mentions a lifelogging system generates a storyline, for which the main challenge is the high semantic difference between raw sensor data and high-level descriptions
	\item \cite{bayindir2017survey} gives as example for bridging raw data to searchable, storyline descriptions:
	\begin{itemize}
		\item automatic text translation for making stories of sequences of wifi scans to describe actions going from place of interest to place of interest
		\item routes around places of interest providing semantic info like the name, the store type, etc, to allow for textual queries (What can I do to allow for queries? What has the research done?)
		\item acceleration data converted to an activity language allowing statistical NLP techniques to create and index activity logs, along with automatic hierarchical segmentation and similar activity retrieval
		\item input organized according to different contexts, story line is created by identifying trigger events and associating those to the context
		\item automatic daily report generator that create basic english sentences from activities of interest
		\item automatic tagging is helpful because they can be used as search terms to retrieve events or information of interest
	\end{itemize}
	\item \cite{bayindir2017survey} mentions task can be implemented as automatic annotation process (your case) in which time is divided in frames and each frame can be assigned one or more semantic tags (!!!!!)
	\item \cite{khan2019smartphone} perform an investigation of the size of data generated by different sensors (no acoustic though). Interesting for checking how they define volume generation checking.
	\item \cite{khan2019smartphone} mentions how continuous sensors are data prone (when combined with a system that always sends the information to a server)
	\item \cite{gurrin2014lifelogging} notes the difference between lifelogging, lifelog and surrogate memory. Lifelog can be stored online or on portable storage device and can be simple like collection of photos to large and complex like accelerometer activity traces. Note this on what to actually store (doesn't mean you have to take this in account, just investigate the difference).
	\item \cite{gurrin2014lifelogging} notes the use cases for life logging as:
	\begin{itemize}
		\item personal healthcare and wellness
		\item Monitoring sleep patterns
		\item Smoking cessation, diet monitoring, sugar intake for diabetics, ...
		\item location logging for social purposes, fitness purposes or general digital diary purposes.
		\item triggered recall memory prosthesis
		\item logging employee activities for legal reasons, replacing manual record taking, logging information access activities or organisational memory (practical in office environments !!!!)
		\item Market research
	\end{itemize}
	\item \cite{gurrin2014lifelogging} notes a shift away from focused quantified self analytics towards the idea of the totality of life experience. These are difficult because there is "no such thing as a normal day in our lives".
	\item \cite{gurrin2014lifelogging} gives the different topics in lifelogging as:
	\begin{itemize}
		\item sensors
		\item software middleware: multimodal needs cleaning and syncing. Data quality is important for business informatics and environmental sensing. Utilizing the reliability of data streams should be researched.
		\item signal processing: to analyze and structure the data. Structured into units called events (like shots in a movie). Subsequently mine for patterns and deviations.
		\item semantic processing: semantic analysis and annotation (!!!). Lifelog is created.
		\item retrieval models: according to 5 r's (recollecting, reminiscing, retrieving information, reflecting and remembering intensions). Desktop interface would need detailed reflection, quantified-self style; mobile would need real-time recollection or retrieval of information.
		
	\end{itemize}
	\item \cite{gurrin2014lifelogging} Notes that if sensors log and record data on-board, they need enough storage space for a few days and when they upload, take advantage of networks or ad-hoc network. Real-time uploading negatively impacts battery life.
	\item \cite{gurrin2014lifelogging} and others metion SenseCam a lot and Nokia lifelog technology as game changers
	\item \cite{gurrin2014lifelogging} mentions mono 22khz audio of 5.840 hours (1 day) to add up to 227.8GB data in a year
	\item \cite{gurrin2014lifelogging} mentions categories of lifelogging tools as:
	\begin{itemize}
		\item Passive visual capture: wearable always-on image recorder
		\item Personal biometrics
		\item Mobile device contexts: using the smartphone to continuously cpature the context using multiple sensors, non visual
		\item communication activities: logging sms messages and such
		\item Data Creation/Access activities: interactions on computers, non-social
		\item Active capture of life' activities: initiated by the user capturing and logging of activity (e.g. blogging)
		\item Environmental Context and Media: Capturing lifelog from environmental sensors (e.g. in a smart home)
		\item Passive Capture audio: identification of activities, events and activity types by audio sensing alone. Examples (Al Masum Shaikh et al. (2008)., Ellis and Lee (2006); Shah et al.
		(2012), Hayes et al. (2004): Persional audio loop, Heittola et al. (2010): audio context classification using audio event histograms)
		\item performance monitoring
	\end{itemize}
	\item \cite{gurrin2014lifelogging} mentions there is no accepted unit of retrieval, it's all dependent on the use case. With audio it's events. Most data exists in forms that are not typically IR retrievable where the concept of relevance or degree of similarity is integral. Events has become accepted practice in lifelogging.
	\item \cite{gurrin2014lifelogging} summarizes structuring lifelog data as:
	\begin{itemize}
		\item Segmenting data into meaningful units. Events are classic, but a starting point, summarization and aggregation techniques can be better than a list of events.
		\item Annotating events with semantic meanings
		\item Access and retrieval , depends on the use case
		\item Multimodal interaction, meaning new technologies like google glass need access
	\end{itemize}
	\item Zacks and Tversky (2001): “an event is a segment of time at a given location that is conceived by
	an observer to have a beginning and an end”.
	\item \cite{gurrin2014lifelogging} mentions a basic canonical higher level concepts have been defined for event categories (e.g. activities => socializing, relaxing,...). Four general categories of cues have been defined as most effective: people, action, object and place. Who, what, where and when has been defined as well
	\item Categorization of events is important.
	\item \cite{gurrin2014lifelogging} mentions date/time are not effective annotation tools, as humans do not remember the exact time, only the cues.
	\item \cite{gurrin2014lifelogging} notes the categories of use cases as:
	\begin{itemize}
		\item Personal lifelogging applications
		\begin{itemize}
			\item Self-monitoring of activities
			\item Memory assistance
			\item Longer-term assisted living
			\item (EXTRA, not mentioned) Workplace safety
		\end{itemize}
		\item Population-based lifelogging applications: infer from the population
		\begin{itemize}
			\item Workplace: capture processe procedures.
			\item market research
			\item Creating family memories from aggregating individual memories
			\item Measuring behaviour in work, social studies, health research
		\end{itemize}
	\item \cite{gurrin2014lifelogging} mentions ethical examples
	\end{itemize}
	\item \cite{ali2019insight} Mentions smartphone baseed lifelogging systems can be classified into two categories: distributed and integrated
	\item \cite{ali2019insight} Problems with distributed smartphone lifelogging systems are:
	\begin{itemize}
		\item Transimssion delay
		\item problems with data uploading where connectivity cannot be ensured
		\item data transmissions deplete battery power
		\item uploading and storing remotely can induce privacy and security issues
	\end{itemize}
	\item \cite{ali2019insight} none available SBL systems support integrated approach, only come close to it like Ubiqlog and Digital Diary.
	\item \cite{ali2019insight} Very little research attention has been paid to pre-processing SBL data.
	\item \cite{jiang2019memento} Build a lifelogging system activated by emotional EEG sensor data shift, logging video audio or image. It decides what modality of logging to use based on calculating the utility and the cost of logging through that modality. The utility calculation is based on the light condition, the mobility level and the noise level. Defines cost by both energy consumption and resource consumption (memory and processing, obtained by offline training) as well as device temperature. Can also dynamically offload computations to smartphone if available. (!!!) Researching the modality as well as the device setup of lifelogging is important and research on its own.
	\item \cite{jiang2019memento} Mentions MyLifeBits as the first bite to the comprehensive lifelogging system, Sensecam as a memory aid proposal using wearable camera, Experience explorer as an contextual sensing and capturing lifelogger on mobile, UbiqLoq as a lightweight framework allowing devs easily create lifelogging application based on it.
	\item \cite{jiang2019memento} Multimodal lifeloggers need triggering mechanisms in stead of constant storage and processing of all sensors.
	\item \cite{rawassizadeh2013ubiqlog} Designed an extendible framework to design lifelog apps for smartphones. 
	\item \cite{rawassizadeh2013ubiqlog} Mention mobile resources as battery usage, CPU usage, memory usage, disk activity and network activity. They used a resource monitoring tool (Mobile application benchmarking based on the resource usage monitoring) to specifically measure battery, CPU and memory usage.
	\item \cite{gurrin2019overview} A competition for designing lifelogging tools according to specific requirements. Tasks were focused on retrieval of lifelog data, activity detection from lifelog data and exploratory tasks for identifying contexts.
	\item \cite{min2018exploring} Illustrate the importance and effect of device placement on the sound signal as well as activity recognition performance.
\end{itemize}

\subsection{Defining lifelogging}

To start, we need to pin down exactly what the concept of lifelogging is and which activities are referred to as such. The definition described by \citet{dodge2007outlines} which was later used by \citet{gurrin2014lifelogging} and \citet{harvey2016remembering} defines lifelogging as \textit{"a form of pervasive computing, consisting of a unified digital record of the totality of an individual’s experiences, captured multi-modally through digital sensors and stored permanently as a personal multimedia archive”}. The form of digital record as well as its content varies greatly, depending on the use-case. 

Another vision of lifelogging is set out by \citet{mears2016virtual} as being an extreme form of blogging, where the content comes from continuously capturing and recording personal data. This work follows the idea of lifelogging data being continuous, which differentiates from the actual capturing being user-initiated. 

Further note should be made on the terminology used in lifelogging literature. The different core elements in the whole lifelogging process were specified in seminal work performed by \citet{gurrin2014lifelogging}:

\begin{itemize}
	\item \textit{Lifelogging} is the activity of passively gathering, processing and reflecting on the digital sensor data. The sensors can be physical or virtual and the sensor data can consist of raw sensor output or a combination of multi-modal sensor output. 
	\item \textit{A Lifelog} is the data stored coming from the lifelogging process. Usually this data includes the original sensor output (e.g. audio recordings), but not necessarily as only the processed information can be stored as well (e.g. the derived number of steps in stead of the accelerometer data in a step counter). The sensor data can be extended with any additional information derived from the data like annotations or combined with other sensory data. Storage can be either online or offline.
	\item \textit{Surrogate Memory} is the lifelog as well as the software to stucture and manage the data. Information retrieval work is done on how to develop retrieval systems to operate on data of this kind and magnitude. It is a digital library focused on maintaining a log of events from life.
\end{itemize}

In this work, the focus lies on the first part: lifelogging for acoustic data. The implied task is to capture and annotate microphone data, with relevant semantic tags that relevantly describes the input. This can be implemented in multiple ways.

 dividing the time into frames with one or more annotations per frame to build a higher-level description of events, activities or context \cite{bayindir2017survey}. 

%What is lifelogging
\begin{itemize}
	\item \citet{gurrin2014lifelogging} notes the difference between lifelogging, lifelog and surrogate memory. Lifelog can be stored online or on portable storage device and can be simple like collection of photos to large and complex like accelerometer activity traces. Note this on what to actually store (doesn't mean you have to take this in account, just investigate the difference).
	\item \citet{bayindir2017survey} mentions task can be implemented as automatic annotation process (your case) in which time is divided in frames and each frame can be assigned one or more semantic tags (!!!!!)
	\item \citet{gurrin2014lifelogging} notes a shift away from focused quantified self analytics towards the idea of the totality of life experience. These are difficult because there is "no such thing as a normal day in our lives".
	\item \citet{dodge2007outlines} "a form of pervasive computing, consisting of a unified digital record of the totality of an individual’s experiences, captured multi-modally through digital sensors and stored permanently as a personal multimedia
	archive”.
	\item \citet{mears2016virtual} Lifelogging to some extent can be considered as blogging taken to the limit where all personal data is continuously captured and recorded.
\end{itemize}

% Kinds of lifelogging
\begin{itemize}
	\item \citet{gurrin2014lifelogging} notes a shift away from focused quantified self analytics towards the idea of the totality of life experience. These are difficult because there is "no such thing as a normal day in our lives".
	\item \citet{gurrin2014lifelogging} mentions categories of lifelogging tools as:
	\begin{itemize}
		\item Passive visual capture: wearable always-on image recorder
		\item Personal biometrics
		\item Mobile device contexts: using the smartphone to continuously cpature the context using multiple sensors, non visual
		\item communication activities: logging sms messages and such
		\item Data Creation/Access activities: interactions on computers, non-social
		\item Active capture of life' activities: initiated by the user capturing and logging of activity (e.g. blogging)
		\item Environmental Context and Media: Capturing lifelog from environmental sensors (e.g. in a smart home)
		\item Passive Capture audio: identification of activities, events and activity types by audio sensing alone. Examples (Al Masum Shaikh et al. (2008)., Ellis and Lee (2006); Shah et al.
		(2012), Hayes et al. (2004): Persional audio loop, Heittola et al. (2010): audio context classification using audio event histograms)
		\item performance monitoring
	\end{itemize}
\end{itemize}

% Completely mobile vs connected with pc
\begin{itemize}
	\item \citet{rawassizadeh2013ubiqlog} enriched data with semantic info straight on the mobile phone
	\item \citet{gurrin2014lifelogging} Notes that if sensors log and record data on-board, they need enough storage space for a few days and when they upload, take advantage of networks or ad-hoc network. Real-time uploading negatively impacts battery life.
	\item \citet{ali2019insight} Mentions smartphone based lifelogging systems can be classified into two categories: distributed and integrated
		\item \citet{ali2019insight} Problems with distributed smartphone lifelogging systems are:
	\begin{itemize}
		\item Transimssion delay
		\item problems with data uploading where connectivity cannot be ensured
		\item data transmissions deplete battery power
		\item uploading and storing remotely can induce privacy and security issues
	\end{itemize}
	\item \citet{ali2019insight} none available SBL systems support integrated approach, only come close to it like Ubiqlog and Digital Diary.
\end{itemize}

\subsection{Designing lifelogging systems}
%Design requirements
\begin{itemize}
	\item \citet{rawassizadeh2013ubiqlog} listed 5 design requirements for lifelogging systems:
	\begin{itemize}
		\item seamless integration into daily life (nope)
		\item resource efficiency (yup, through analysis)
		\item security (nope, but can be discussed)
		\item long-term digital preservation (nope, but can be discussed)
		\item information retrieval (needs investigation -> efficient tagging/indexing coming from system)
	\end{itemize}
	\item \cite{bayindir2017survey} mentions a good lifelogging system should operate during the day without user attention, which utilizes phone resources, which has an effect on power consumption.
	\item \cite{bayindir2017survey} mentions a lifelogging system generates a storyline, for which the main challenge is the high semantic difference between raw sensor data and high-level descriptions
	\item \cite{jiang2019memento} Multimodal lifeloggers need triggering mechanisms in stead of constant storage and processing of all sensors.
\end{itemize}

% CHallenges
\begin{itemize}
	\item \cite{bayindir2017survey} mentions a lifelogging system generates a storyline, for which the main challenge is the high semantic difference between raw sensor data and high-level descriptions
	\item \cite{khan2019smartphone} mentions how continuous sensors are data prone (when combined with a system that always sends the information to a server)
	\item \cite{gurrin2014lifelogging} notes a shift away from focused quantified self analytics towards the idea of the totality of life experience. These are difficult because there is "no such thing as a normal day in our lives".
	\item \cite{gurrin2014lifelogging} Notes that if sensors log and record data on-board, they need enough storage space for a few days and when they upload, take advantage of networks or ad-hoc network. Real-time uploading negatively impacts battery life.
	\item \cite{ali2019insight} Problems with distributed smartphone lifelogging systems are:
	\begin{itemize}
		\item Transimssion delay
		\item problems with data uploading where connectivity cannot be ensured
		\item data transmissions deplete battery power
		\item uploading and storing remotely can induce privacy and security issues
	\end{itemize}
	\item \cite{min2018exploring} Illustrate the importance and effect of device placement on the sound signal as well as activity recognition performance.
\end{itemize}

% Ethical, security and privacy challenges
\begin{itemize}
	\item \cite{gurrin2014lifelogging} mentions ethical examples
\end{itemize}

%Mobile Device specificities (mobile challenges)
\begin{itemize}
	\item \cite{khan2019smartphone} perform an investigation of the size of data generated by different sensors (no acoustic though). Interesting for checking how they define volume generation checking.
	\item \cite{gurrin2014lifelogging} mentions mono 22khz audio of 5.840 hours (1 day) to add up to 227.8GB data in a year
	\item \cite{rawassizadeh2013ubiqlog} Mention mobile resources as battery usage, CPU usage, memory usage, disk activity and network activity. They used a resource monitoring tool (Mobile application benchmarking based on the resource usage monitoring) to specifically measure battery, CPU and memory usage.
\end{itemize}

% What are the steps in lifelogging?
\begin{itemize}
	\item \cite{gurrin2014lifelogging} gives the different topics in lifelogging as:
	\begin{itemize}
		\item sensors
		\item software middleware: multimodal needs cleaning and syncing. Data quality is important for business informatics and environmental sensing. Utilizing the reliability of data streams should be researched.
		\item signal processing: to analyze and structure the data. Structured into units called events (like shots in a movie). Subsequently mine for patterns and deviations.
		\item semantic processing: semantic analysis and annotation (!!!). Lifelog is created.
		\item retrieval models: according to 5 r's (recollecting, reminiscing, retrieving information, reflecting and remembering intensions). Desktop interface would need detailed reflection, quantified-self style; mobile would need real-time recollection or retrieval of information.
	\end{itemize}

	\item \cite{gurrin2014lifelogging} summarizes structuring lifelog data as:
	\begin{itemize}
		\item Segmenting data into meaningful units. Events are classic, but a starting point, summarization and aggregation techniques can be better than a list of events.
		\item Annotating events with semantic meanings
		\item Access and retrieval , depends on the use case
		\item Multimodal interaction, meaning new technologies like google glass need access
	\end{itemize}
	\item \cite{ali2019insight} Very little research attention has been paid to pre-processing SBL data.
	\item \cite{rawassizadeh2013ubiqlog} Designed an extendible framework to design lifelog apps for smartphones. 
\end{itemize}

% Trigerring lifelogging
\begin{itemize}
	\item \cite{jiang2019memento} Build a lifelogging system activated by emotional EEG sensor data shift, logging video audio or image. It decides what modality of logging to use based on calculating the utility and the cost of logging through that modality. The utility calculation is based on the light condition, the mobility level and the noise level. Defines cost by both energy consumption and resource consumption (memory and processing, obtained by offline training) as well as device temperature. Can also dynamically offload computations to smartphone if available. (!!!) Researching the modality as well as the device setup of lifelogging is important and research on its own.
	\item \cite{jiang2019memento} Multimodal lifeloggers need triggering mechanisms in stead of constant storage and processing of all sensors.
\end{itemize}

% Making lifelog data retrievable
\begin{itemize}
	\item \cite{bayindir2017survey} gives as example for bridging raw data to searchable, storyline descriptions:
	\begin{itemize}
		\item automatic text translation for making stories of sequences of wifi scans to describe actions going from place of interest to place of interest
		\item routes around places of interest providing semantic info like the name, the store type, etc, to allow for textual queries (What can I do to allow for queries? What has the research done?)
		\item acceleration data converted to an activity language allowing statistical NLP techniques to create and index activity logs, along with automatic hierarchical segmentation and similar activity retrieval
		\item input organized according to different contexts, story line is created by identifying trigger events and associating those to the context
		\item automatic daily report generator that create basic english sentences from activities of interest
		\item automatic tagging is helpful because they can be used as search terms to retrieve events or information of interest
	\end{itemize}
	
	\item \cite{bayindir2017survey} mentions task can be implemented as automatic annotation process (your case) in which time is divided in frames and each frame can be assigned one or more semantic tags (!!!!!)
	\item \cite{gurrin2014lifelogging} mentions there is no accepted unit of retrieval, it's all dependent on the use case. With audio it's events. Most data exists in forms that are not typically IR retrievable where the concept of relevance or degree of similarity is integral. Events has become accepted practice in lifelogging.
	\item Zacks and Tversky (2001): “an event is a segment of time at a given location that is conceived by
	an observer to have a beginning and an end”.
	\item \cite{gurrin2014lifelogging} mentions a basic canonical higher level concepts have been defined for event categories (e.g. activities => socializing, relaxing,...). Four general categories of cues have been defined as most effective: people, action, object and place. Who, what, where and when has been defined as well
	\item Categorization of events is important.
	\item \cite{gurrin2014lifelogging} mentions date/time are not effective annotation tools, as humans do not remember the exact time, only the cues.
\end{itemize}


\subsection{Examples of life logging}

% General notes
\begin{itemize}
	\item Not a lot of work on purely accoustic lifelogging. Most is on visual or digital data. Closest is multi-modal data.
	\item No Lifelogging polyphonic events
	\item 
\end{itemize}

%Acoustic lifelogging
\begin{itemize}
	\item iRemember
	\item lifleogging: archival and retrieval
	\item Kapture
	\item Minimal-impact audio-based personal archives, (2006 Ellis and Lee) Accessing minimal-impact personal audio archives
	\item The personal audio loop: Designing a ubiquitous audio-based memory aid
	\item Automatic lifelogging:A novel approach
	\item Audio context recognition
	\item prof-lifelog
	\item Speech/Music Indexing
	\item bodyscope
\end{itemize}

%Multimodal (with acoustic)
\begin{itemize}
	\item Digital diary
	\item Pensieve
	\item Mobile Life-logger
	\item Sensor fusion tech : C. tomas, Sensor Fusion and its Applications, Stanley Sensor fusion
	\item Kern et al. 2007
\end{itemize}


% Use cases of life logging
\begin{itemize}
	\item \cite{gurrin2014lifelogging} notes the use cases for life logging as:
	\begin{itemize}
		\item personal healthcare and wellness
		\item Monitoring sleep patterns
		\item Smoking cessation, diet monitoring, sugar intake for diabetics, ...
		\item location logging for social purposes, fitness purposes or general digital diary purposes.
		\item triggered recall memory prosthesis
		\item logging employee activities for legal reasons, replacing manual record taking, logging information access activities or organisational memory (practical in office environments !!!!)
		\item Market research
	\end{itemize}

	\item \cite{gurrin2014lifelogging} notes the categories of use cases as:
	\begin{itemize}
		\item Personal lifelogging applications
		\begin{itemize}
			\item Self-monitoring of activities
			\item Memory assistance
			\item Longer-term assisted living
			\item (EXTRA, not mentioned) Workplace safety
		\end{itemize}
		\item Population-based lifelogging applications: infer from the population
		\begin{itemize}
			\item Workplace: capture processe procedures.
			\item market research
			\item Creating family memories from aggregating individual memories
			\item Measuring behaviour in work, social studies, health research
		\end{itemize}
		\item \cite{gurrin2014lifelogging} mentions ethical examples
	\end{itemize}
\end{itemize}

% Examples of lifelogging
\begin{itemize}
	\item \cite{gurrin2014lifelogging} and others metion SenseCam a lot and Nokia lifelog technology as game changers
	\item \cite{jiang2019memento} Build a lifelogging system activated by emotional EEG sensor data shift, logging video audio or image. It decides what modality of logging to use based on calculating the utility and the cost of logging through that modality. The utility calculation is based on the light condition, the mobility level and the noise level. Defines cost by both energy consumption and resource consumption (memory and processing, obtained by offline training) as well as device temperature. Can also dynamically offload computations to smartphone if available. (!!!) Researching the modality as well as the device setup of lifelogging is important and research on its own.
	\item \cite{jiang2019memento} Mentions MyLifeBits as the first bite to the comprehensive lifelogging system, Sensecam as a memory aid proposal using wearable camera, Experience explorer as an contextual sensing and capturing lifelogger on mobile, UbiqLoq as a lightweight framework allowing devs easily create lifelogging application based on it.
	\item \cite{rawassizadeh2013ubiqlog} Designed an extendible framework to design lifelog apps for smartphones. 
	\item \cite{gurrin2019overview} A competition for designing lifelogging tools according to specific requirements. Tasks were focused on retrieval of lifelog data, activity detection from lifelog data and exploratory tasks for identifying contexts.
	\item \cite{gurrin2014lifelogging} Passive Capture audio: identification of activities, events and activity types by audio sensing alone. Examples (Al Masum Shaikh et al. (2008)., Ellis and Lee (2006); Shah et al.
	(2012), Hayes et al. (2004): Persional audio loop, Heittola et al. (2010): audio context classification using audio event histograms)
\end{itemize}








\section{Acoustic Event Detection with Deep Learning}

\subsection{Notes}
\begin{itemize}
	\item Label Strength
	\item Learning approaches (deep or nah)
	\item Characteristics of deep learning in aed
	\item Types of deep learning
	
\end{itemize}

\textbf{Multi-task AED}








