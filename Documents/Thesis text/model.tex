\chapter{System Design}

% Typing datasets
% Typing tasks
% Typing combining methods


\section{Assumptions}

\begin{itemize}
	\item For feature extraction, the device can hold each entire extracted dataset separately in memory, but no 
	\item 
\end{itemize}


TODO: Explain how the system was modeled based on the design requirements

\begin{figure}[h]
	\centering
	\includegraphics[width=\textwidth]{../../../../Documents/TU Delft/Thesis 2/System Explanation/SimplifiedOverview}
	\caption{Simplified System overview}
\end{figure}


\section{Easy changeable variables}

\subsection{Different datasets}

\subsection{Different Sample Rate}

\subsection{Different Feature Extraction}

\subsection{Different Data Transformation}

\subsection{Different Dataloading}

\subsection{Different DL Model}

\subsection{Different Optimizer}

\subsection{Different loss calculation}

\subsection{Different loss combination}

\subsection{Different Stopping Criteria}

\subsection{Different Saving Locations}

\section{Easy expansions}

\subsection{Adding Datasets}

\subsection{Adding Tasks to datasets}

\section{Simplifying abstractions}

\subsection{Saving/Reading Extracted Datasets}

\subsection{Index Mode}

\subsection{Combination of different datasets}
TODO: The ConcatTaskDataset function

\subsection{Train/test generation}

\subsection{Evaluation}

\subsection{Result Saving and Visualizing}

\subsection{Interrupted Learning}

\section{Developmental side rails}

\subsection{Abstract Data Reader}

\subsection{Abstract Extraction Method}

\subsection{Standardized valid input}

\subsection{Centralized Train/test Operations}