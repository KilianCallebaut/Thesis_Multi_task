\chapter{Problem Statement}
TODO: Explain that this chapter is about defining the problem and what the solving system should be

\section{Use Cases}
TODO: Explain the need for requirements by clarifying examples



\section{Developers}

\subsection{Researchers}
TODO: Who need to vary different parts of the pipeline and report on their effect

\subsection{Optimizers}
TODO: Who need to be able to build and train the best performing model

\section{Design Principles}

TODO: Outline the assumptions you make that the system is built on and the objectives the framework has to achieve to offer better developmental support

\subsection{Fast Prototyping}

Deliver a framework which users can use to quickly build, train and test a deep learning multi-task pipeline without compromise. 

\subsection{Dynamic Handling Of Deviating Cases}

The framework should be capable of handling a wide range of variations in datasets and task structures, without requiring adjustments to be made to the overall structure.

\subsection{Easy Extendibility}

Every part of the pipeline that has variations, should be variable in a modular way.

\subsection{Abstracted File Management}

Functionalities that require a system to write or read files on the system, should be abstracted to the point that users should not be forced to input more than the desired location of the files.

\subsection{Guided Development}

Developers getting familiar with the system should both be in possession of working examples, as well as guiding functions they have to implement in order to build a functioning pipeline.

\section{Non-functional Requirements}

\section{Functional Requirements}