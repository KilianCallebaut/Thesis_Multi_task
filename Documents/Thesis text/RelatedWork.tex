\chapter{Related Work}

\section{Audio Recognition}
TODO: Explain the field of audio classification, how it normally works, what kind of tasks there are and what kind of things are researched

\section{Multi-task Learning}
TODO: Explain the field of multi-task learning, where it came from, what the paradigm brought in improvements and what kind of things are researched
TODO: Explain multi task learning with unimportant auxiliary task for performance improvement

\section{Multi-task Deep Learning Audio Tasks}
TODO: Explain the merging of these fields and what (little) work has been done there so far. Also what it requires for more work to be done and what the current work lacks.

Audio recognition tasks have only recently seen the adoption of multi-task frameworks, but for a quickly expanding set of reasons. This section maps out the key applications where, why and how deep learning systems where created to address problems. Single task focused systems were the way to go as instinctively it made sense that the best performances would be obtained by focusing on improving solely on building a system addressing that problem. Two factors however changed this notion. For one, systems in real life contexts often have to perform multiple related tasks at once TODO: REFERENCE. For another, sharing information between related tasks have been shown to improve performances as it averages noise patterns and creates a more robust representation of the original data for the final classifier(s) TODO: REFERENCE. Here, it will be demonstrated how these factors influenced the recent evolution towards trying to build systems that optimise for multiple objectives at once. 

TODO: create table of papers

Where it was used:
Multiple tasks that needed to be performed but were done without multi-task
prof-life log Ziaei 2013
Zhang et al 2019
Park 2020

\textbf{Demonstrating Necessities.} What will be done first is demonstrate the context where multiple tasks need to be performed, but were not optimized at the same time in a multi-task setting.  

Speech recognition tasks
Lu et al 2004
Seltzer 2013
Dong et al. 2015
Kim et al. 2017
sakti et al 2016
Meyer
“Speech enhancement
and recognition using multi-task learning of long short-term memory recurrent neural networks chen


Shift towards Multi-Task as performance improver

- Same dataset auxiliary
lee 2019
abrol 2020
Deshmukh 2020
zeng et al 2019
Fernando 2020


- Other dataset other task
Sakti et al 2016
Tonami 2019
xu 2019
Huang 2020
Imoto
Jung et al 2020
komatsu 2020


- Same Task split in two
Morfi 2018
Pankajakshan
Phan et al 2019
Xia 2020


Multi-Task Learning for other purposes
Georgiev
Tagliasacchi




\section{Development Frameworks}
TODO: Get examples in from other development frameworks, how they answered the needs in their fields and why they are needed