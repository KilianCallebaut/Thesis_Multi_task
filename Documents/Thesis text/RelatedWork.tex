\chapter{Related Work} \label{RelatedWork}

\section{Audio Recognition}
%TODO: Explain the field of audio classification, how it normally works, what kind of tasks there are and what kind of things are researched

Audio contains a rich amount of information. As one of the more important senses, audio provides humans with perceptual information about their environment and its occupants. Machines can analyse audio in the same way, in search of performing numerous functions humans do simultaneously. These tasks include for example deriving semantic information like recognizing speech \citep{hannun2014deep}, contextual information like recognizing the scene \citep{abesser2020review} or entity identification like recognizing a piece of music \citep{huang2019state}. Audio classification is important for the field of pattern recognition, finding an ever expanding amount of applications. \\

Audio data and its relating tasks have been grouped and divided using numerous definitions, which change the focus and structure of the systems built. For one, the sound data can be classified according to domain groups. \cite{duan2014survey} does this by subdividing them into human voice sounds, artificial sounds and natural sounds. Human voice includes speech, coughs and singing. Artificial sounds refer to human activity based sounds like traffic, aircraft and music. Natural sounds then include animals, weather and nature sounds. However within the same document, they utilise another distinction between sound data namely music, speech and environmental sounds. The point is that applications have been developed which target different definitions of sound collections that are targeted. Grouping music together in the same domain with other human activity noise makes sense for applications that only need to identify a sound in an audio fragment as music as in \cite{park2020augmenting}, but not when the exact music piece needs to be identified in the presence of background noise \citep{huang2019state}.\\

Audio recognition tasks thus refer to the automated annotating of audio data for various sounds and/or sound groups present within. A typical \textbf{pipeline} for audio recognition tasks goes as follows: \\

\begin{enumerate}
	\item Microphone records raw audio and stores it as a sampled time series. 
	\item Preprocessing is applied to accentuate certain properties in the audio signals, the choice of which is often dependant on the eventual application and what the system needs to be able to differentiate within the signals \citep{georgiev2017heterogeneous}. 
	\item These time series then are divided into either overlapping or non-overlapping frames, for which each frame a feature gets calculated. The features of all the frames are then stored as a collection which represents a single data input instance. This is also called the \textbf{feature matrix}. A comprehensive overview of features are given in \cite{mitrovic2010features}
	\item The feature matrices are transformed in context of the other matrices, augmenting the extracted feature representations or scaling the values to the same specified value range. 
	\item The collections of data instances then form inputs for classification or regression problems. A number of instances are used for training a neural network which learns a representation of the data that it can optimally use for deriving the correct labels.
\end{enumerate}

There are two distinctions in audio based learning tasks: instance-level and frame-level tasks. The difference between these is if the task needs to produce labels for the whole audio fragment or for events present within an audio fragment. The latter is a complexer task than the former.  Instance level tasks include Acoustic Scene Recognition, Audio Tagging, Speaker Identification and Emotion Recognition. Frame level tasks include Acoustic Event Detection and Automatic Speech Recognition. When speaking about frame level learning, the task involves temporal localization of the individual labels. For that reason the task is usually defined as a classification task for each mini segment, but detecting the presence of an event and determining when it is present can also be formulated as determining its start and stop times. Intuitively, this thus means that these sort of tasks can be split into two separate ones, even without defining it as a multi-task classification network. \\

Labels in instance based recognition work pretty straight forward. The data contains audio clips for which one or multiple labels are linked, depending if multiple classes can be present at once. A system thus takes a whole sound clip, extracts its features and outputs the corresponding labels the clip is an instance of. Labels in frame based recognition are different. As stated, clips are subdivided in frames and the output of the system is a series of labels, one for each frame which corresponds to a specific time step. \\

%TODO: Give deeper explanations of example fields and how they work (AED, ASC, ASR)
%TODO: Explain the type of research done in the field

%Audio recognition tasks are investigated for a plethora of goals and contexts. The most straight forward 

\section{Multi-task Learning}
%TODO: Explain the field of multi-task learning, where it came from, what the paradigm brought in improvements and what kind of things are researched
%TODO: Explain multi task learning with unimportant auxiliary task for performance improvement

% What is multi-task learning
Multi-task Learning is a learning paradigm that performs inductive transfer by sharing representational knowledge from multiple related tasks. The principle goal of Multitask Learning is stated as being the improvement of generalization performance \citep{caruana1997multitask}. To explain the meaning of this description and objective, the explanation of the paradigm will be given incrementally.\\

To begin, the term task must be defined, in order to understand what will be combined. A task is a collection of data instances and corresponding target labels, combined with a function mapping those instances to targets. Target labels can be represented as one-hot vectors for classification or continuous valued vectors for regression purposes \citep{meyer2019multi}. Learning a task infers learning the parameters for the mapping function to optimally link the instances to the known true labels. Usually, mapping functions are learned by only utilising the task's data and targets, which is also referred to as learning a representation. Large problems are broken into small independent subproblems that are learned this way separately in parallel and recombined afterwards. This is counterproductive as in real world problems, information from different task representations can not be used by the other tasks \citep{caruana1997multitask}. The result might be that the representation is overfitted to the utilised data, which falters when applied in the real world due to factors that weren't present in the data the representation is trained on. It is often found that task systems' performances degrade significantly when there is a discrepancy between the training conditions and testing conditions \cite{lu2004multitask}. \\

% How does multi-task learning work
The response to these shortcomings is multi-task learning, which learns a shared representation on multiple tasks at the same time to a degree. The mapping function's parameters are optimized for multiple objectives at once. Only one model is produced for which there are task specific parameters and shared task parameters. The specific parameters are updated using the error signals from a single task, while the shared ones are updated using the signals from all tasks \cite{meyer2019multi}. Multi-task models following this definition, can be built in a huge amount of ways, with the main concern being what parameters are shared which relates to what level of abstraction two tasks need to share in order to build a better representation for both. There are also two forms of sharing parameters: hard-sharing and soft-sharing. In the former, the shared model parameters for all tasks are the exact same. In soft-sharing settings, parameters are shared more loosely, with parameter updates happening only for one task, but the distance between the different tasks' parameters being regularized for their distance. This forces the parameters to be similar \citep{ruder2017overview}. This work mainly focuses on the hard parameter sharing setting. \\

% How mtl works in deep learning
For deep learning neural networks, like convolutional neural networks and recurrent neural networks, this parameter sharing set-up takes the form of sharing layers. The output of each layer is treated as a shared feature representation for the subsequent ones \citep{zhang2018overview}. This can be taken as a direct transformation from the input or combine hidden representation from multiple tasks to form more powerful hidden representations when different data sources require different sources. \\

% What does a simple multi-task model work like
\begin{figure}
	\centering
	\includegraphics[width=0.3\linewidth]{"../../../../Documents/TU Delft/Thesis 2/System Explanation/examplemultihead.drawio"}
	\caption{Example multi-head neural network model}
	\label{fig:examplemultihead}
\end{figure}

To illustrate what this looks like, the simplest model structure, the multi-head model, is given as an example in figure \ref{fig:examplemultihead}. This demonstrates the situation where shared layers build a robust internal hidden representation utilising knowledge from multiple tasks in a feed-forward deep neural net. Combining dataset representations to a shared layer can be used to extend a single task's data sources. Multiple output heads can be defined for one dataset for a representation optimized for multiple purposes. The illustration performs both. Learning wise this achieves inductive transfer of knowledge and provides an inductive bias. Bias makes the model prefer some hypotheses over others \cite{caruana1997multitask}. Defining extra targets can provide more control over the working of the model. Traditionally additional tasks are added in order to improve robustness, but recently more applications are investigated for additional improvements, like fairness \citep{oneto2019taking}, compression \citep{georgiev2017heterogeneous} and expanding limited strong labelled data with weaker data \cite{lee2019label}.\\

% What gets researched in multitask learning
%% Model forms
%% Auxiliary tasks ->
%% training set-ups -> multi instance, teacher student model
%% when does it work -> 



\section{Multi-task Deep Learning Audio Tasks}

Audio recognition tasks have only recently seen the adoption of multi-task frameworks, but for a quickly expanding set of reasons. This section maps out the key applications where, why and how deep learning systems where created to address problems. Single task focused systems were the way to go as instinctively it made sense that the best performances would be obtained by focusing on improving solely on building a system addressing that problem. Two factors however changed this notion. For one, systems in real life contexts often have to perform multiple related tasks at once \cite{ziaei2013prof} . For another, sharing information between related tasks have been shown to improve performances as it averages noise patterns and creates a more robust representation of the original data for the final classifier(s) \citep{xia2019multi}. Here, it will be demonstrated how these factors influenced the recent evolution towards trying to build systems that optimise for multiple objectives at once. An overview of the used papers is given in tables \ref{table:combinations}, \ref{table:combinations2} and \ref{table:combinations3} \\

\begin{table}[ht]
	\caption{Tried Combinations} % title of Table
	\centering % used for centering table
	\begin{tabular}{p{0.2\textwidth}p{0.6\textwidth}p{0.2\textwidth}} % centered columns (4 columns)
		\hline\hline %inserts double horizontal lines
		Title & Tasks & Classifier   \\ [0.5ex] % inserts table
		%heading
		\hline % inserts single horizontal line
		\citet{lu2004multitask} & Automatic Speech Recognition; Speech Enhancement; Gender Detection & RNN \\ \hline
		\citet{seltzer2013multi} & Phone Label Recognition; State context recognition; Phone context recognition & DNN-HMM \\ \hline
		\citet{panchapagesan2016multi} & Keyword Spotting; Large Vocabulary Continuous Speech Recognition Senones Targets Recognition & DNN \\ \hline
		\citet{sakti2016deep} & Automatic Speech Recognition; Acoustic Event Detection & DNN \\ \hline
		\citet{georgiev2017heterogeneous} \citet{georgiev2017low} & Speaker Identification; Emotion Detection; Stress Detection; Acoustic Scene Classification & DNN \\ \hline
		\citet{kim2017speech} & Emotion Detection; Auxiliary tasks: Arousal Level; Valence Level; Gender Detection & CNN \\ \hline
		\citet{nwe2017convolutional} & Acoustic Scene Classification (Grouped scenes as different tasks) & CNN \\ \hline
		\citet{phan2017dnn} & 	Detection error; distance error; optimization confidence &  DNN \\ \hline
		\citet{sun2017compressed} & Keyword Spotting; Large Vocabulary Continuous Speech Recognition Phone Targets Recognition & DNN-HMM \\ \hline
		\citet{kremer2018inductive} & Word Error Rate and Character-Level Automatic Speech Recognition & CNN \\ \hline
		\citet{morfi2018deep} & Audio Tagging; Event Activity Detection & DNN \\ \hline
		\citet{lee2019label} & Main Tasks: Audio Tagging; Speaker Identification; Speech Command Recognition (Keyword Spotting); Auxiliary Tasks: Next-Step prediction; Noise Reduction; Upsampling & DNN \\ \hline
		\citet{lopez2019keyword} & Keyword Spotting; Own-voice/External Speaker Detection & DNN\\ \hline
		\citet{meyer2019multi} & Speech/Noise detection; Language Identification & CNN + LSTM \\ \hline
	\end{tabular}
	\label{table:combinations} % is used to refer this table in the text
\end{table}

\begin{table}[ht]
	\caption{Tried Combinations (Continued)} % title of Table
	\centering % used for centering table
	\begin{tabular}{p{0.2\textwidth}p{0.6\textwidth}p{0.2\textwidth}} % centered columns (4 columns)
		\hline\hline %inserts double horizontal lines
		Title & Tasks & Classifier   \\ [0.5ex] % inserts table
		%heading
		\hline % inserts single horizontal line
		\citet{pankajakshan2019polyphonic} & Sound Activity Detection (Event Activity Detection); Sound Event Detection (Audio Tagging) & CRNN \\ \hline
		\citet{phan2019unifying} & 	Detection error; distance error; optimization confidence & CNN \\ \hline
		\citet{tonami2019joint} & Acoustic Event Detection; Acoustic Scene Classification & CRNN for AED, CNN for ASC  \\ \hline
		\citet{xia2019multi} & Acoustic Event Type Detection (Audio Tagging); Predict frame position information (Event Activity Detection) & CNN \\ \hline
		\citet{xu2019multi} & Acoustic Event Detection; Acoustic Scene Classification & \\ \hline
		\citet{zeng2019spectrogram} (1) & Emotion Detection; Music/Speech Classification & DNN \\ \hline
		\citet{zeng2019spectrogram} (2) & Accent Recognition; Speaker Identification & DNN \\ \hline
		\citet{abrol2020learning} & Fine and Coarse Labels Acoustic Scene Classification & DNN \\ \hline
		\citet{deshmukh2020multi} & Acoustic Event Detection; Reconstruct Time Frequency Representation of Audio &  CNN \\ \hline
		\citet{fernando2020temporarily} & Acoustic Event Type Detection (Audio Tagging); Predict Frame Position Information (Event Activity Detection) & LSTM \\ \hline
		\citet{huang2020guided} & Audio Tagging; Temporal Detection (Event Activity Detection) & CNN PT/PS model \\ \hline
		\citet{huang2020multi} & Audio Tagging; Event Boundary Detection (Event Activity Detection) & CNN \\ \hline
		\citet{imoto2020sound} & Acoustic Event Detection; Acoustic Scene Recognition & CNN \\ \hline
		\citet{jung2020acoustic} & Acoustic Scene Recognition; Audio Tagging & DNN \\ \hline
		\citet{komatsu2020scene} & Acoustic Event Detection; Acoustic Scene Recognition & CNN \\ \hline
			\end{tabular}
	\label{table:combinations2} % is used to refer this table in the text
\end{table}

\begin{table}[ht]
\caption{Tried Combinations (Continued)} % title of Table
\centering % used for centering table
\begin{tabular}{p{0.2\textwidth}p{0.6\textwidth}p{0.2\textwidth}} % centered columns (4 columns)
\hline\hline %inserts double horizontal lines
Title & Tasks & Classifier   \\ [0.5ex] % inserts table
%heading
\hline % inserts single horizontal line

		\citet{tagliasacchi2020multi} & Keyword Spotting; Speaker Identification; Language Identification; Music/Speech Classification; Bird Audio Detection; Urban Acoustic Scene Classification; Music Instrument Pitch Detection; Music Instrument Detection & CNN  \\ \hline
		\citet{wu2020domain} & Keyword Spotting; Domain Prediction & CNN + LSTM \\
		[1ex] % [1ex] adds vertical space
		\hline %inserts single line
	\end{tabular}
	\label{table:combinations3} % is used to refer this table in the text
\end{table}

\textbf{Non-Parallel Multiple Tasks.} What will be done first is demonstrate the context where multiple tasks need to be performed, but were not optimized at the same time in a multi-task setting. For painting a clearer picture of contexts where multiple tasks have to be learned for the same data, this work will first delve into the context where this has been addressed without a multi-task framework. In TODO:REFERENCE, a lifelogging system is designed which tries to annotate the naturalistic audio from a sensor device with different types of labels to create a contextual summary of a person's day. This takes an audio stream and performs speech activity detection, the results of which then in turn uses for its other tasks. It estimates the amount of speakers in an audio segment; uses that estimate to then recognize the speakers and determine the primary and secondary speakers. At the same time it also performs the task of "environmental sniffing" or detecting the current environment of the device. The results of this set-up are illustrated in TODO: GET DIAGRAM. This example is given to illustrate the need and the types of tasks that would be learned at the same time in a context. Task information is shared here, but segmentally in stead of parallel. The thing is however that this does not share information to improve the representation, which ignores potentially useful information. This also requires every task to be performed individually, a set-up which might not always be allowed in terms of time and resource constraints.\\

To clarify how this singular focus forms the basis for problems in naturalistic audio contexts, the work done by \cite{park2020augmenting} is addressed. Here, the context of an audio device is being detected in a short time frame of seconds, by recognizing events that are tied together in wat they call contexts. First of all is the obvious need in this case for fast detection. Second point lies in their results. They find that including the detection of speech events significantly reduces the performance of the general classifier's performance. Speech audio is different in structure from other background audio events and likely requires its own model for reliable detection.\\

With the first example being one of tasks connected segmentially and the second being unified in the same classifier, the stage is set for exploring the compromise between the two: separated, yet parallel task inference. While Multi-Task settings had demonstrated their use in visual tasks TODO: REFERENCE, the adoption in audio tasks was (similar to the trajectory of deep learning algorithms) a bit slower to develop. The first domain that showed its potential were speech recognition tasks.\\


\textbf{Multi-Task Speech Recognition Tasks.} For Automatic Speech Recognition, multi-task learning has been around for a while. These tasks accept windows of audio features as input and return posterior probability distributions over phonetic targets \citep{meyer2019multi}. The targets can range from whole words to phonetic parts and even simply characters. The multi-task model itself either serves as an end-to-end recognition model or performs a modeling function in a hybrid model. \\

\cite{lu2004multitask} proposed an architecture where noisy speech audio is fed into a Recurrent Neural Network that has a number of shared layers with three output heads. One for the prediction of words, one for enhancing speech and one for gender detection. With adding these auxiliary tasks on the same dataset, they succesfully managed to improve the original task's - Speech recognition - performance. Their reasoning for utilising multi-task learning was that performance degrades dramatically for when there is a mismatch between train and testing conditions. Multi-Task learning added robustness to classification performance. \\

A lot of the work done in multi-task speech recognition focuses on this sort of improvement of the original signal's representation through adding additional optimization goals. \cite{seltzer2013multi} focus on the improvement of phoneme recognition by adding the prediction of the phone labels, state contexts and phone contexts - context here being predicting the label in the previous and next time frame. \\

\textbf{General purpose audio} The techniques that were found for Speech Recognition purposes eventually found their way for general purpose audio tasks. There are a few trends within acoustic multi-task definitions that will be discussed here, which will be important bases for the system to cover.  The trends deal with how the additional task was utilised in the same model. Four of these trends are addressed in this section: Auxiliary tasks on the same dataset, combining tasks from different datasets, splitting a task and finally multi-task learning for non-performance related issues. \\

\textbf{Auxiliary tasks on the same dataset.} The first strain of set-ups concern audio tasks that receive an additional task on the same dataset. This is usually with the aim to improve the original task's performance. These in turn come in two forms: supervised tasks \citep{kim2017speech} \citep{zeng2019spectrogram} \citep{abrol2020learning} \citep{fernando2020temporarily} \citep{wu2020domain} \citep{sun2017compressed} \citep{lopez2019keyword} \citep{panchapagesan2016multi} and self-supervised tasks \citep{lee2019label}  \citep{deshmukh2020multi} \citep{pankajakshan2019polyphonic} \citep{lu2004multitask}.  \\

Often, the auxiliary task is not important and only serves to help the neural network create a representation of the data that is formed for its qualities the additional tasks require. With a supervised task as an auxiliary task, this takes a more semantic form. The task is semantically related. In \cite{zeng2019spectrogram}, the multi-task set-up results are explored for combining speaker identification and accent recognition in one case and emotion recognition in speech and emotion recognition in song in another. In both these instances, the semantic connections for the targets are clear.  \cite{abrol2020learning} does this concept slightly different and learns basically the same task but at different abstraction levels, in order to improve learn leveraging hierarchical relation structures. The set-up for \cite{fernando2020temporarily} contains detecting speech activity along with predicting the next audio segment, using layers of LSTM. Here layers  are shared at different levels, with a generative adversarial network that learns the loss function.\\


For the other scenario where the set-up adds a self-supervised task, this usually comes in the form of transforming the input signal according to different qualities, which will be transferred to the task at hand. \cite{lee2019label} investigates the effect of adding next-step prediction, noise reduction and upsampling of an audio signal as an auxiliary task to audio tagging, speaker identification and keyword detection tasks. Each possible subgroup of auxiliary tasks are tried.  With this it successfully tries to get more efficiency out of its labelled data. \cite{deshmukh2020multi} recreates the Time Frequency representation of audio as its auxiliary task to event detection. This aims to reduce the noise of the internal representation used for classification which makes it function better as a classifier based on noisy recordings (like in real life contexts). It also utilises Multiple Instance Learning, which is a learning mechanism where instances get put into positive and negative bags.  Positive bags do not only contain instances that are positive for that label - at least one for sure -, but negative bags do. \cite{lu2004multitask}, which has been discussed before, is a hybrid of these two forms. One of its auxiliary task is gender detection (first scenario) and the other is speech enhancement (second scenario). \\

For the system to be created this signifies a lot of different possibilities in terms of possible targets, beyond simple labels as well as a possibly inexhaustible range for models and training requirements. Any abstraction made in this regard must be fully adaptable. Tasks should also freely be addable or retractable in order to investigate its combinations to the main task. Furthermore, grouping instances should be possible, as seen in the example for Multiple Instance Learning. \\


\textbf{Different datasets, different tasks.} Another utilisation of multi-task set-ups is to bring together two different tasks with - usually - two different datasets. The idea is to bring two tasks together that could prove to be beneficial for each other as successfully performing their combined task is directly beneficial for the task at hand. The improvement is either aimed for one task like in the previous case, or it is to benefit the performance of both tasks at once. \\

One of the clearest and more popular examples of this set-up is in set-ups that combine the Acoustic Event Detection (AED) task - which detects which sound events are present at each time frame - and the Acoustic Scene Recognition (ASR) - which detects in which background scene a sound sample takes place in \citep{tonami2019joint} \citep{xu2019multi} \citep{imoto2020sound} \citep{jung2020acoustic} \citep{komatsu2020scene}. The idea is that information on detecting specific scenes will help in detecting events, either by learning the noise pattern related to that scene or because certain events are inherently linked to certain scenes. \cite{tonami2019joint} does exactly this, building a simple multi-head model that shares three layers before venturing off in task specific networks. The labels which improved in this context are investigated, finding that labels which are only connected to each other but not other labels in the parallel task improve significantly in accurate detection. \cite{xu2019multi} performs this but with a sole focus on improving AED. \cite{imoto2020sound} builds on this but reforms the ASR to output soft labels (percentages in stead of deterministic labels). For training there is a separate independent network for ASR that teaches the ASR task in the multi-head model in what is a called a teacher-student learning framework. Finally \citep{jung2020acoustic} and \citep{komatsu2020scene} adopt a model, where the different task results are directly combined afterwards to only output better labels for one task.\\

Other examples are limited and underdeveloped.  The work done by \cite{sakti2016deep} might give a hint for the reason. They tried to bring together speech recognition with environmental sound detection, yet have to end up limiting the shared layers in both tasks, mainly finding that combined features improve the set-up slightly. These two tasks might have been too unrelated for the multi-task set-up to offer improvement, but very little investigation into this aspect was performed. The final example is found in the work by \cite{huang2020guided} and \cite{huang2020multi}. This is interesting as it finds a way to improve its intended task (AED) with weaker labeled data. It combines its model at different layers with splitting branches, adding an optional side branch for stronger labeled data that improves overal performance if available. This potential for extending original datasets with weaker datasets but with likely more instances is immense, which was proven by winning first place in the 2019 DCASE Task 4 competition. This shows how the multi-task framework has been gaining momentum over recent years, with the capacity to improve available task information, whether it is by simply providing more contextual information or finding a way to provide more valuable training data.\\


\textbf{Same task split.} Present in the last example \cite{huang2020guided} is the idea of using a multi-task framework to split a complex task in two separate tasks and combining the results for a single improved prediction. This mostly happens for AED tasks, by redefining this task as two tasks: determining the event type and determining the time . This either happens through splitting the task into a classification task for the type and a sound activity detection task that simply outputs whether any sound event is present in a time frame \citep{morfi2018deep} \citep{pankajakshan2019polyphonic}. The other way it has been performed was by adding a regression task for the time offset of events to add more exact information on when the time exactly starts. These always involve some sort of result fusion after (probabilistic) prediction output. \\

In \cite{phan2017dnn} and \cite{phan2019unifying} also resembles this model, but is only a multi-task framework right at the end, as it optimizes for different criteria. One is the detection error, one is the distance error of events and the final is the confidence in the first two predictions. This does not completely defines the task as different ones, simply optimizes for different criteria, but still requires fusion after the fact. \\

The final interesting case is that in \cite{nwe2017convolutional}. This splits up ASR into multiple tasks, which are all still ASR but in different recording conditions. The labels are split up into three groups: Indoor scenes, sparse outdoor scenes and crowded outdoor scenes, each group being defined as its own task. \\

In essence tasks on datasets can be reconstructed in an infinite amount of ways. The system should offer this level of control over datasets and tasks. With the idea in \cite{huang2020guided}, it should also take in account parallel models that influence the training of the multi-task framework that was built.\\


\textbf{Multi-Task Learning For Other Reasons.} Finally, there are the cases where the multi-task set-up is used for reasons that do not relate to performance improvements of any kind. This idea is still limited, but illustrates how much further the multi-task framework's applicability goes. These investigate the capabilities of combining datasets and tasks that are unrelated and possibly require combining a large amount of heterogeous tasks to be combined together, compared to the previous trends.\\

In \cite{georgiev2017heterogeneous}, a multi-task framework is used as a means to compress deep learning models that have to perform different tasks. Even in the instances where independent models are technically more optimal, it might be preferable to combine their network layers. This is useful if the model has to be deployed on computationally constrained devices. Multiple independent models can require too high resource costs, so the argument is that using a multi-task framework that does not sacrifice too much in terms of performance compresses the amount of complexity for execution of the same tasks.\\

For the same reasons, the work in \cite{tagliasacchi2020multi} is performed, but offers more adaptation to the different tasks. Both of these examples bring together numerous tasks and datasets. These require a lot of work on combining dataset differences for execution in the same network. Also take in account that there is a process to arrive at these models, in which design and parameter differences have to be varied (often for each dataset in the same way), evaluated and compared with the variations before. \\

What is taken away from each of these trends, is that there are numerous opportunities in acoustic multi-task learning. An huge amounts of variations and possibilities still have to be explored. Facilitating free experimentation with the differences in tasks and the way to combine these would be crucial for promoting further research in these fields. A platform could be built that can dynamically handle these cases while offering abstractions that quickly deal with physical problems that can arise from these cases. \\

Summarizing the take aways from observing these trends goes as follows. In the first case, datasets can have multiple tasks defined on them. Extra tasks come in different forms themselves, even as aggregated forms from other tasks (multiple instance learning). In the second case, different tasks and different datasets get combined. This of course necessitates dealing with dataset differences while the different ones can be seen as one big dataset or not. Models outside the multi-head model can also be brought in to affect training. The performance effect on the datasets and the output labels also needs options to be evaluated closely. The third case clarifies that developers need a lot of control over both tasks and datasets. The final case demonstrates that multi-task networks are implemented for more than performance improvements and a possibly huge amount of datasets and tasks can be combined together. Handling of these are as likely to be different for each case as they are to be the same for all.\\

% Evaluation


\section{Development Frameworks}
%TODO: Get examples in from other development frameworks, how they answered the needs in their fields and why they are needed
%TODO: Apparently there

Frameworks offer designs and pieces of code that are reusable and functionally allow the creation of different applications within its domain. These code structures are generic, intended to reduce the cost of development. The flexibility of frameworks are hard to design compared to specific applications as a lot of possibilities and abstractions have to be planned for beforehand. Framework design is its own subject that has a lengthy research history already. \\

\cite{schmid1997systematic} and \cite{roberts1996evolving} are early works discussing the characteristics of framework development. The main element that has to be designed for is variability. Software patterns have to be put in to place that are organized in two parts \citep{ben2004uml}: hot-spots and the core. Hot-spots are places in a system where implementing applications have their own specific adaptation in place. The core is common to all applications derived from the framework. \\

\begin{figure}
	\centering
	\includegraphics[width=0.49\linewidth]{"../../../../Documents/TU Delft/Thesis 2/System Explanation/BlackBox"}
	\includegraphics[width=0.49\linewidth]{"../../../../Documents/TU Delft/Thesis 2/System Explanation/WhiteBox"}	\caption[Black White Box Hot Spot]{Black box hot-spot (left) and White box hot-spot (right)}
	\label{fig:blackbox}
\end{figure}

Hot-spots come in two forms: black box and white box. Black box hot-spots have their variations predefined and implementations can only pick one. White box hot-spots require a developer's own implementation by programming a class or subsystem. These are illustrated in figure \ref{fig:blackbox}.\\

Variability itself comes with a few characteristics \citep{schmid1997systematic}. 
\begin{itemize}
	\item The common responsibility which overcouples different alternatives.
	\item different alternatives that realize the common responsibility
	\item The variability type that depends on the subjects structure.
	\item The multiplicity of alternatives and the structure of alternatives
	\item The point in time where the alternative is picked and implemented (fixed or at run-time)
\end{itemize}

UML design of a framework has to visualize the flexibility and points of variability clearly. This means that some extensions are necessary compared to the usual standards for normal applications. This is a subject which has been researched in a number of papers \citep{bouassida2001uml} \citep{ben2004uml}. The main takeaway is that the points of extension should be clear for developers immediately. Designing of a framework requires abstraction from specific implementations and how this happened in this work will be expanded upon in section \ref{Design:Framework}. \\

No prior work is found on offering a framework for multi-task deep learning, let alone for audio purposes. Closely related is \cite{dcaserepo2021}, a tool released by the time behind the yearly dcase \cite{dcase} competition for audio recognition tasks. This is a standalone library which offers tools for processing audio and metadata files. The designs are mostly based around offering containers which come with functionalities for processing audio and metadata. However, this adds a lot of extra layers to the data which is a problem for verifying different datasets are processed in the same way. This also separates target handling from the inputs, which forms a problem for operations that rely on their connection (e.g. filtering). Training functionalities are present but need to fit in the predefined mold and do not allow multiple different datasets (without having to redefine them as exactly the same dataset). Finally, this requires that data is stored in similar ways to their own datasets which is not always the case. All these elements contribute to the overlying problem for multi-task set-ups that deals with collections of datasets, for which this implementation the handling of data is too individualistic. \\

Deep learning development libraries like \cite{dong2017tensorlayer} \cite{tokui2019chainer} and \cite{chen2019selene} are often based on larger libraries like PyTorch \cite{paszke2019pytorch} and TensorFlow \cite{tensorflow}. \cite{nguyen2019machine} offers a great overview of a large number of frameworks and deep learning libraries. Their views on aiding development of deep learning usually boils down to optimisation and avoiding problems in specific specialised fields. Looking at \cite{dong2017tensorlayer} the framework presented numerous modules for efficiently executing advanced deep learning  models, while also optimising data loading  and providing intermediary caching functionalities for fast execution. Furthermore, they add a dataset module which not only optimises storage and retrieval but facilitates management through indexing operations. There even is a wholistic workflow module where the task execution is optimized and defined by a simple insertion of an input dataset, an output dataset and a model. However - ignoring how these singular dataset definitions are a roadblock for multi-task work - this approach lacks the exploratory nature of research and development, where parameters and steps often are replaced in order to find the most fitting system. \\

\cite{chen2019selene} demonstrates how a lot of smaller libraries work, where the development improvement is focused on a particular field, in this case genomic sequences. This does however demonstrate some of the same characteristics this work is interested in when defining developmental speed up. Knowing the types of inputs it will receive, it greatly facilitates the process of adding new datasets to existing architectures created with the system. Also included are immediate visualizations, while minimizing the entire process in two modules: model creation and command line interface for executing prototypical use cases. This work argues that this comes closer to envisioning its usefulness in research settings, due to its openness and anticipation of bringing in new data, as well as providing ready to go functionalities for the research process like visualization.\\

\cite{tokui2019chainer} offers the tools to improve the research cycle, but focuses its functionalities on creating models using improved definitions which does seem to provide for a more wholistic view of designing and implementing deep learning pipelines (as opposed to the usual approach where model definition and the eventual training run are separate). Again this is done to open up optimisation possibilities. Also returning is the lack of variability without having to dismantle the whole implementation and of course the lack of multi-task support. At this point it is good to talk what makes the multi-task paradigm "break" this framework. The issue in this instance is due to the fact that multi task learning requires updates which are based on multiple losses. When calculating the loss functions, some instances in a batch might or might not be part of an a particular task and only the related loss to that task must be calculated and used in the update of the model. While a response to this might be to separate training of the model per task, this limits performance of multi-task pipelines which benefit greatly from the case when the model is trained for multiple tasks at the same time \cite{georgiev2017low}.\\

What is gathered from these deep learning frameworks is that there is simply a gap of support for multi-task pipelines. On top of that is there a lack of variability within the envisioned pipelines, often due to their focus on performance. While in no way is this a slight against these works, it does arguably signify that even if multi-task support would be provided, the added amount of unnecessary double work it would require to transmit functions to each branch every time a change is required introduces dreadful, error-prone implementation work. A lot of these frameworks are great in case it is simply about implementation of a design, but this is not likely to be the case. On the other hand, there are some great ideas present in these works, like the caching from \cite{dong2017tensorlayer}, a wholistic view of the pipeline which compresses the entire pipeline to a single definition phase that opens up opportunities for optimisation in \cite{tokui2019chainer} and the research based command line tools from \cite{chen2019selene}.\\ 


One framework that this work did take inspiration from however was \cite{de2012deap}, a development framework for evolutionary learning. This is a white-box framework that has a lot of parallel necessities in functionalities while also having to take in account the amount of possibilities that still need to be allowed by extension.\\

Finally, this work is an extension of PyTorch \citep{paszke2019pytorch}, an imperative stile deep learning library. Building on their groundworks, this work offers abstractions to their deep learning modeling and training functionalities to better the development of multi-task networks alongside it. A number of their design principles are kept in place like the pythonic style, the focus on researchers and pragmatic performance.\\


